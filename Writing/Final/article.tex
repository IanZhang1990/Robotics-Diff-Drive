\title{ Determine Paths in the Same Homotopy Class }
\author{
        Yinan Zhang\\
        Department of Computer Science\\
        Dartmouth College\\
        Hanover, New Hampshire 03755, \underline{US}
        \and
        Devin Balkcom \\
        Department of Computer Science\\
        Dartmouth College\\
        Hanover, New Hampshire 03755, \underline{US}
}
\date{\today}

\documentclass[11pt]{article}

\begin{document}
\maketitle

\begin{abstract}
Describing the  \ldots
\end{abstract}

\section{Introduction}
This is time for all good men to come to the aid of their party!

\section{Related Work}\label{related work}
In this section, we first discuss the basics of sampling-based motion planning and the method of sampling on medial axis. With the information of positions and their clearance in c-space, we have samples as spheres. We then discuss the dual shapes of the union of balls which plays an important role in understanding the shape of c-space and its topology. Some works on graph theory and topology will finally be introduced. 

\begin{description}
\item[A] \emph{Sampling-based Motion Planning} \hfill \\

\item[B] \emph{Medial Axis Sampling} \hfill \\

\item[C] \emph{Dual Shapes of Unions of Balls} \hfill\\

\item[D] \emph{Betti Numbers} \hfill \\

\item[E] \emph{Homotopy Class} \hfill \\

\end{description}

\section{Our Method}\label{method}

Traditionally, sampling based algorithms will build a graph for configuration space. These kind of graphs is proved to be able to find near optimal path, but can hardly be used to analysis the topology of paths. We want to to build a stronger represent of c-space that is both able to cover optimal path and study the topology structure of c-space. In this section, we are trying to answer the question of "are two paths in the same homotopy class". We will first give the outline of our algorithm, and introduce each phase step by step. 

\begin{description}

\item[A] \emph{The Algorithm} \hfill \\

\item[B] \emph{Sampling in C-Space} \hfill \\

\item[C] \emph{Constructing Hypergraph of C-Space} \hfill \\

\item[D] \emph{Breaking A Graph into Loop-free Parts} \hfill\\

\item[E] \emph{Determine Paths Homotopy Class} \hfill \\

\end{description}

\section{Experiments}\label{experiments}
We now describe the some experiments and their results.

\section{Conclusions and Future Work}\label{conclusions}
We worked hard, and achieved very little.

\bibliographystyle{abbrv}
\bibliography{main}

\end{document}